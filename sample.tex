\documentclass[
  a4paper,
  fleqn,
  student, % alternatively: instructor
  solution, % alternatively: nosolution
  narrow, % alternatively: nonarrow. 
  narrowfactor=.7,% narrowfactor=x \in [0,1], default=.7
    ]{exercisesheet}

% Setting up the fonts.

\usepackage[T1]{fontenc}
\usepackage[utf8]{inputenc}

\usepackage{libertine}
\usepackage[sc]{mathpazo}

\def\sheettitlefont{\sffamily\fontsize{36}{40}\selectfont}
\def\sheetsubtitlefont{\sffamily\Huge\color{maincolor}}
\def\sheetdeadlinefont{\sffamily\LARGE}
\def\sheetlecturefont{\sffamily\footnotesize}
\def\sheetsemesterfont{\sffamily\footnotesize}
\def\sheetlecturerfont{\sffamily\footnotesize}
\def\sheetexercisefont{\small\sffamily}
\def\sheetsubexercisefont{\footnotesize\sffamily}

\usepackage[dvipsnames]{xcolor}
\colorlet{maincolor}{BurntOrange!70!black}


\usepackage[%
  pdfborder={0 0 0},
    ]{hyperref}

\usepackage[english]{babel}
\usepackage{blindtext}

\lecture{From the \emph{Foo} to the \emph{Bar}}
\lecturer{Prof.~Dr.~Foo~Bar}
\semester{Winter~Semester~2011/2012}
\student{John~Doe}

\begin{document}

\maketitle

\tableofcontents

\pagestyle{headings}

\sheet[
    title=In the Beginning was the \emph{Foo},
    deadline={1.~September~2011}
  ]

This exercise sheet serves as a sample for the usage of the
\exercisesheet{} class. As there is currently no manual for this
package, we ask you to take a look at the source code of this sample
for details on the usage of this package. \hfill \emph{Thank you.}

\exercise[
    title=This is an Exercise,
    credits=4,
    label={first_exercise}
  ]

\blindtext 

\subexercise[
    title=This is a Subexercise,
    credits=1,
  ]

\blindtext

\subexercise[
    title=This is another Subexercise,
    credits=3,
  ]

\blindtext

\begin{solution}[
    exerciselabel={first_exercise}
  ]

  \blindtext

\end{solution}
  

\sheet[
    title=Yet another Exercise Sheet,
    number=12,
    deadline={2.~September~2011}
  ]

\end{document}
