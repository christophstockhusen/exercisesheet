\documentclass{report}

\usepackage[T1]{fontenc}
\usepackage[utf8]{inputenc}

\usepackage{libertine}
% \usepackage[sc]{mathpazo}
\def\ttdefault{txtt}

\usepackage[dvipsnames]{xcolor}
\colorlet{maincolor}{BurntOrange!70!black}

\usepackage[english]{babel}

\usepackage[
  a4paper,
  hmargin={2cm,2cm},
  vmargin={3cm,3.5cm}
    ]{geometry}

\usepackage{enumitem}
\setlist{nolistsep}

\usepackage{listings}
\lstset{
    basicstyle=\ttfamily,
    keywordstyle=\color{maincolor},
    language=[LaTeX]TeX,
    frame=lines,
    backgroundcolor=\color{maincolor!20},
    framerule=.8pt,
    numbers=left,
    numberstyle=\footnotesize
  }

\def\exercisesheet{\texttt{exercisesheet}}

\usepackage{titlesec}
\titleformat{\chapter}[display]{\sffamily\huge}{Chapter~\thechapter}{1ex}{}
\titleformat{\section}[display]{\sffamily\large}{Section~\thesection}{1ex}{}

\usepackage[
  pdfborder={0 0 0},
    ]{hyperref}

\begin{document}

\begin{titlepage}
  \null \vskip .2\textheight
  \fontsize{36}{40}\selectfont
  \sffamily Reference Manual for the \par
  \exercisesheet{} Class \par \bigskip \bigskip
  \Large \color{maincolor}%
  Christoph Stockhusen \par \bigskip
  \today \vskip .1\textheight
\end{titlepage}

\begin{abstract}
  The \exercisesheet{} class provides a convenient way to produce
  hiqh-quality exercise sheets for both students and instructors. This
  is the official reference manual. It contains tutorials for students
  and instructors that explain the first steps in using this class.
  Moreover, a complete list of the commands defined by the
  \exercisesheet{} class is contained.
\end{abstract}

\tableofcontents

\chapter{Introduction and Tutorials for Students and Instructors}

My aim of programming the \exercisesheet{} class is to provide both
studends and teachers a \LaTeX{} class for easily typesetting
exercisesheets and their solutions. The following sections contain two
tutorials: The first tutorial aims at students, the second at
teachers.

\section{A Tutorial for the Student Lia}

This semester, Lia, student of computer science, participates in the
course \emph{Theoretical Computer Science}. Her tutor encourages her
to typeset the exercise sheets with \LaTeX{} as this would be a good
training for her bachelors thesis. Moreover, he recommends the
\exercisesheet{} class as it provides a nice layout and many useful
macros that might help Lia on her first steps with \LaTeX{}.

After Lia managed to install a commen \LaTeX{} distribution she
immediately starts to create her first exercise sheet, as the deadline
is already in two days. She creates a new file called
\lstinline{TCS.tex} and types the following lines to load the document
class.

\begin{lstlisting}
\documentclass[
  a4paper,
  student,
    ]{exercisesheet}
\end{lstlisting}


\section{A Tutorial for the Instructor Prof.~John~Doe}

\chapter{Reference}

This section provides a comprehensive list of macros and options that
are provided by the \exercisesheet{} class.

\section{Document Class Options}

\section{Macros}

\end{document}
