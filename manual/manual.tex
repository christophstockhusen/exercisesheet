\documentclass{report}

\usepackage[T1]{fontenc}
\usepackage[utf8]{inputenc}

\usepackage{libertine}
% \usepackage[sc]{mathpazo}
\def\ttdefault{txtt}

\usepackage[dvipsnames]{xcolor}
\colorlet{maincolor}{BurntOrange!70!black}

\usepackage[english]{babel}

\usepackage[
  a4paper,
  hmargin={2cm,2cm},
  vmargin={3cm,3.5cm}
    ]{geometry}

\usepackage{enumitem}
\setlist{nolistsep}
\setitemize{label={--}}

\usepackage{listings}
\lstset{
    basicstyle=\ttfamily,
    keywordstyle=\color{maincolor},
    language=[LaTeX]TeX,
    frame=lines,
    backgroundcolor=\color{maincolor!20},
    framerule=.8pt,
    % numbers=left,
    % numberstyle=\footnotesize,
    morekeywords={
      sheet,
      exercise,
      subexercise,
      title,
      number,
      label,
      exerciselabel,
      exercisesheet,
      student,
      maketitle,
      tableofcontents,
      lecture,
      lecturer,
      semester,
      colorlet
    },
    otherkeywords={a4paper}
  }

\def\exercisesheet{\texttt{exercisesheet}}

\usepackage{titlesec}
\titleformat{\chapter}[display]{\sffamily\huge}{Chapter~\thechapter}{1ex}{}
\titleformat{\section}[display]{\sffamily\large}{Section~\thesection}{1ex}{}
\titleformat{\subsection}[display]{\sffamily}{Subsection~\thesubsection}{1ex}{}

\usepackage[
  pdfborder={0 0 0},
    ]{hyperref}

\begin{document}

\begin{titlepage}
  \null \vskip .2\textheight
  \fontsize{36}{40}\selectfont
  \sffamily Reference Manual for the \par
  \exercisesheet{} Class \par \bigskip \bigskip
  \Large \color{maincolor}%
  Christoph Stockhusen \par \bigskip
  \today \vskip .1\textheight
\end{titlepage}

\begin{abstract}
  The \exercisesheet{} class provides a convenient way to produce
  hiqh-quality exercise sheets for both students and instructors. This
  is the official reference manual. It contains tutorials for students
  and instructors that explain the first steps in using this class.
  Moreover, a complete list of the commands defined by the
  \exercisesheet{} class is contained.
\end{abstract}

\tableofcontents

\chapter{Introduction and Tutorials for Students and Instructors}

My aim with programming the \exercisesheet{} class is to provide both
students and teachers a \LaTeX{} class for easily typesetting
exercisesheets and their solutions. Therefore, the following two
sections contain tutorials for students as well as teachers for using
the \exercisesheet{} class. Files with the complete code of the
templates are available with this package; You can download them from
\url{http://exercisesheet.googlecode.com}.

\section{A Tutorial for the Student Lia}

This semester, Lia, student of Computer Science, plans to typeset the
solutions of all the exercise sheets with \LaTeX{}. In the last
semesters she wrote her solutions per hand which had several
drawbacks:
\begin{itemize}
  \item Often her first solutions to exercises contained little errors
    that she had to correct. Very small errors were easy to correct,
    but many times she had to insert some sentences or formulas into
    her solution with the result that Lia had to write several pages
    of solutions again.
  \item One week 
  \item \ldots
\end{itemize}
Lia thinks that she can overcome these drawbacks by writing her
solutions at her computer with \LaTeX. As this will be Lia's first
experience with \LaTeX{} she asks her tutor Bo for help. He feels
happy about Lia's plans because he thinks that this will be a good
preparation for her bachelor's thesis. Bo recommends the
\exercisesheet{} class that he often used for typesetting his
solutions for exercise sheets.

At home from university, Lia downloads and installs a common \LaTeX{}
distribution and starts to create a simple template file that she
plans to use for her next exercise sheets. She creates a file 
\lstinline[language=bash]{template_student.tex} and opens it in her
favorite text editor \emph{GVim}. She sets up her file with the
following lines.
\lstinputlisting[linerange={1-9}]{template_student.tex}
Lia passes the \lstinline{a4paper} to the
\lstinline|\documentclass{exercisesheet}| command because she wants
the paper layout to be \textsc{din}~A~4. Moreover, she uses the
\lstinline{student} option which makes the class print its titles in
an appropriate manner for submitting the solutions to her instructor.
Then Lia specifies the name of the lecture, the name of the lecturer,
the semester, and her name. Within these commands the makes use of the
tilde~\lstinline{~} in order to suppress undesirable line breaks, like
Donald Knuth recommended in the \TeX{}book.

With this setup Lia is now ready to typeset her first exercise sheet.
Hence, she types in the following.
\lstinputlisting[linerange={21-22,27-29,64-65}]{template_student.tex}
Lia uses the \lstinline|\sheet| command to start a new exercise sheet.
The command requires no argument but accepts an optional list of
key-value pairs that affect the appearance of the sheet. She decides
to use the keys \lstinline{title} to set a temporary title.

That was easy! At this point nothing can stop her typesetting her
first solution. Lia continues typing. She uses the following lines to
typeset a solution for an exercise that is split into two
subexercises.

\lstinputlisting[linerange={32-46}]{template_student.tex}

Lia is really happy about the simplicity of the \exercisesheet{}
class. She notices that the can typeset multiple exercise sheets
within a single \LaTeX{} document by simply using the
\lstinline{\sheet} macro repeatedly. Therefore, she asks herself
whether the \exercisesheet{} class is capable of typesetting a
title page and a table of contents. She tries whether the following
standard \LaTeX{} macros will work.

\lstinputlisting[linerange={23-23,25-25}]{template_student.tex}

After two runs a table of contents appears in her document that lists
the exercise sheets and the exercises to her complete satisfaction.
But now, as she has compiled her document taken a look at the produced
\textsc{pdf} file, she is not very happy with the default fonts of
\LaTeX{}. She decides to use some other fonts and bring more color
into her sheets. Therefore, she loads the \lstinline{xcolor} package,
defines a nice orange as her base color, sets the serif font to
Palatino and the sans-serif font to Biolinum. Furthermore, she
redefines the fonts of the sheet title and the subtitle. This also
affects the title page and now Lia is really happy with her result.
\lstinputlisting[linerange={11-19}]{template_student.tex}

Altogether, Lia has written the following template code.

\lstinputlisting{template_student.tex}


\section{A Tutorial for the Instructor Prof.~John~Doe}

\chapter{Reference}

This section provides a comprehensive list of macros and options that
are provided by the \exercisesheet{} class.

\section{Document Class Options}

\section{Macros}

\end{document}
